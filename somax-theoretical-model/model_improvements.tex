\chapter{More Control Parameters}\label{sec:6-model-improvements}
A number of new features and other architectural improvements have been added to the main runtime framework to improve the quality of the output as well as give the user a higher degree of parametric control. This section will briefly expand upon the notation presented in section \ref{sec:3-generate} and then present each of the newly implemented features.

%%%%%%%%%%%%%%%%%%%%%%%%%
\section{Some Additional Notation for Peak Scaling}\label{ssec:runtime-recap}

The process of peak scaling was described in section \ref{sec:3-generate-scale} and has not been altered apart from what was described in section \ref{sec:5-transforms} of this report. However, to better understand the concepts introduced in this chapter, some further definitions will be necessary. 

Formally, we can, based on equation \ref{eq:3-peak-scaling-generic}, describe the individual scaling of a peak matrix $\bm P_w = \begin{bmatrix} \bm p^{(1)} & \dots & \bm p^{(K)} \end{bmatrix}$, $\bm P_w \in \mathbb R^{3\times K}$ at time step $w$ as a function $\Gamma \colon \mathbb R^{3\times K} \rightarrow \mathbb R^{3\times K}$ but with a number of side effects. For all of the operations presented in this section, it is however easier to describe this behaviour in terms of a scale operation of individual peaks, i.e. $\Gamma \colon \mathbb R^3 \rightarrow \mathbb R^3$ for each peak $\bm p^{(k)} \in \bm P_w$, $\bm p^{(k)} \in \mathbb R^3$ for $k = 1,\dots,K$. This function is at time step $w$ defined as 
	\begin{align}
		\bm{p^{(\ast)}} =  \Gamma\left(\bm p, t^{(\mathcal Y)}_w, \mathcal S^{(\mathcal C)}_{\bm p}, \mathcal H_w^{(q)}\right), \qquad \forall\, \bm p \in \bm P_w
	\end{align}
	where $t^{(\mathcal Y)}_w \in \mathbb R_+$ denotes the (scheduler's) time at generation time step $w$, $\mathcal S^{(\mathcal C)}_{\bm p}$ denotes the slice in the corpus corresponding to the time $t^{(\mathcal Y)}$ of the peak (per equation \ref{eq:peak-transform}) and $\mathcal H_w^{(q)}$ is the history of output slices from the previous $q \in \mathbb Z_+$ time steps, i.e. 
	\begin{equation}
		\mathcal H_w^{(q)} = \left\lbrace S^{(\mathcal Y)}_{w-1},\dots,S^{(\mathcal Y)}_{w-q}\right\rbrace,
	\end{equation}
	where $\mathcal S^{(\mathcal Y)}_{w-m}$ denotes the generated output $m$ time steps ago.
	
	
This behaviour was previously handled at each level in the architecture by the \texttt{MergeAction}, which handled both merging peaks from multiple layers as well as scaling them individually, but has now been moved to the \texttt{ScaleAction} class and is only computed once at the top level (\texttt{Player} class)\footnote{See \cite{somaxsoftware2021} for context regarding these different classes}. Do note that there are no limitations to the number of scale actions that can be applied, so the final value of each peak $\bm  p^{(\ast)}$ can for $J$ scale actions be defined as
	\begin{align}
		\bm p^{(\ast)} = \left(\Gamma^{(J)} \circ \Gamma^{(J-1)} \circ \dots \circ \Gamma^{(1)} \right)\left(\bm p, t^{(\mathcal Y)}_w, \mathcal S^{(\mathcal C)}_{\bm p}, \mathcal H_w\right)
	\end{align}
	where $\circ$ denotes function composition. Each of the scale actions described in the following sections are optional and can be added/removed dynamically by the user.

%%%%%%%%%%%%%%%%%%%%%%%%%
\section{Transposition Consistency}
One concern related to the introduction of transpositions is that all transforms are given equal probability. In certain (tonal) cases, this may mean that unbiased and too frequent jumping between transpositions will cause issues between the original harmonic function of a particular chord and its harmonic function in the new real-time context. The Transposition Consistency scale action was introduced as an attempt to remedy this by simply scaling all peaks with a user controlled parameter $\gamma \in \mathbb R_{[0, 1]}$ if the peak's transposition does not correspond to the transposition of the previous output slice, i.e.
	\begin{align}
		\bm p^{(\ast)} = \left\lbrace\begin{array}{cl}
				\bm p & \text{ if } \mathcal T_{w-1} = \mathcal T_{\bm p}\\
				\begin{bmatrix} 
					1 & 0 & 0 \\
					0 & \gamma & 0\\
					0 & 0 & 1
				\end{bmatrix} \bm p & \text{ otherwise}
			\end{array}\right.
	\end{align}
	where $\mathcal T_{w-1}$ denotes the transform of the previous output slice $\mathcal S^{(\mathcal Y)}_{w-1}$ and $\bm p$ is defined (according to equation \ref{eq:peak-transform}) as
	\begin{align}
		\bm p = \begin{bmatrix} 
			t^{(\mathcal C)}_{\bm p} \\ y_{\bm p} \\ \mathcal T_{\bm p} 
		\end{bmatrix}.
	\end{align}


%%%%%%%%%%%%%%%%%%%%%%%%%
\section{Taboo and Auto-Jump}
Taboo and auto-jump are two behaviours that existed in earlier versions of Somax but were not originally included when the code base was rewritten in \cite{borg_2019}. With this update, they have been reintroduced as scale actions in the new architecture. Taboo is a simple but efficient way to prevent a player from getting stuck in a short loop by reducing the score $y_{\bm p}$ of all peaks whose temporal coordinates $t_{\bm p}^{(\mathcal C)}$ corresponds to the $q \in \mathbb Z_+$ (user-controlled parameter) most recent output slices to 0, i.e.
	\begin{align}
		\bm p^{(\ast)} = \left\lbrace\begin{array}{cl}
				\begin{bmatrix} 
					t^{\mathcal C)}_{\bm p} \\ 0 \\ \mathcal T_{\bm p}
				\end{bmatrix} 
				& \text{ if } \mathcal S^{(\mathcal C)}_{\bm p} \in \mathcal H_w^{(q)}  \\ 
				\bm p & \text{ otherwise}.
			\end{array}\right.
	\end{align}
	
The purpose of the auto-jump scale action is to prevent too long sequences of the original corpus to be replicated in the output by gradually increasing the probability of a jump. More precisely, it will gradually decrease the score of peaks corresponding to the in the corpus following slice within a user-defined interval $[a, b], a, b \in \mathbb Z_+, b > a$ number of generated output events by a factor $\gamma \in \mathbb R_{[0,1]}$ calculated according to the following formula:
	\begin{align}
			\gamma = \left\lbrace\begin{array}{ll}
				1 & \text{ if } \tau \le a\\
				0.5^{\tau - a} & \text{ if } a < \tau < b\\
				0 & \text{ otherwise},
			\end{array}\right.
	\end{align}
	where $\tau$ corresponds to the number of in the corpus consecutive slices computed by algorithm \ref{alg:tau}, where the function $u\left(\mathcal S^{(\mathcal Y)}\right)$ returns the index $u\in[1,\dots,U]$ of $\mathcal S^{(\mathcal Y)}$ in its original corpus $\mathcal C$.
		\begin{algorithm}[h!]	
			\caption{Computing number of consecutive slices $\tau$}
			\begin{algorithmic}\label{alg:tau}
				\STATE{$\tau := 0$}
				\STATE{$u := u\left(\mathcal S^{(\mathcal Y)}_{w-1}\right)$}
				\STATE{$i := 2$}
				\WHILE{$u - u\left(\mathcal S^{(\mathcal Y)}_{w-i}\right) = 1$}
					\STATE{$\tau := \tau + 1$}
					\STATE{$u := u\left(\mathcal S^{(\mathcal Y)}_{w-i}\right)$}
					\STATE{$i := i +1$}
				\ENDWHILE
			\end{algorithmic}
		\end{algorithm}
The result of this calculation is only applied to the score of any peak corresponding to the in the corpus next consecutive slice, i.e.
	\begin{align}
		\bm p^{(\ast)} = \left\lbrace\begin{array}{cl}
				\begin{bmatrix} 
					t^{(\mathcal C)}_{\bm p} \\ \gamma y_{\bm p} \\ \mathcal T_{\bm p}
				\end{bmatrix} 
				& \text{ if } u\left(\mathcal S^{(\mathcal C)}_{\bm p}\right) = u\left(\mathcal S^{(\mathcal Y)}_{w-1}\right) - 1  \\ 
				\bm p & \text{ otherwise}.
			\end{array}\right.
	\end{align}


%%%%%%%%%%%%%%%%%%%%%%%%%
\section{Parametric Filters}
Somax has up to this point mainly been operating on two musical dimensions when selecting its output: chroma and pitch. Here, a number of scale actions are introduced to give the user some degree of control over other musical dimensions such as loudness, spectral distribution and various temporal aspects of the generated output.  Similarly to other scale actions, all of these filters operate by scaling the score of the peak according to a parameter $\gamma \in \mathbb R_{[0,1]}$, i.e.
	\begin{align}\label{eq:filter-scaling}
		\bm p^{(\ast)} = \begin{bmatrix}
			t^{(\mathcal C)} \\ \gamma y_{\bm p} \\ \mathcal T_{\bm p}
		\end{bmatrix}
	\end{align}
	where $\gamma$ can be described as a function operating on a trait $\theta$  defined in the corpus. More specifically, each of those filters (apart from the last one) take the form of a gaussian defined so that
	\begin{align}
		\gamma = \gamma\left(\theta,\bar{\theta}\right) = \text{exp}\left[-\frac{(\theta-\bar{\theta})^2}{2s^2}\right]
	\end{align}
	where $\bar{\theta}$ (unless otherwise specified) defines the desired value for the targeted musical trait and $s$ (in all cases) defines the desired spread of the targeted parameter, effectively controlling how steeply peaks corresponding to slices whose traits fall outside the requested value should be attenuated. Both $\bar{\theta}$ and $s$ are (unless otherwise specified) user-controlled parameters. As Somax cannot construct corpora from audio files at the moment, all of these filters are only defined for corpora based on MIDI files.\footnote{but similar solutions exist in the audio domain and the architecture fully supports adding such solutions, should this change in the future.} Note that none of these scale actions will remove any peaks no matter how far away their corresponding traits fall from the requested value, they will at most drastically reduce their probability. The idea behind this is to to give the user control over the overall direction, where occasional fluctuations occur, ideally resulting in a more dynamic output. Also note that if no peaks exist for a given time step $w$, the scale action will not have any impact on the resulting output.
	
	
\subsection{Loudness}
The Loudness scale action allows the user to specify a requested maximum velocity within a slice. The parameters for $\gamma$ are defined as
	\begin{align}
		\theta = \text{max}\left( \left\lbrace \frac{\mathcal N\texttt{.velocity}}{128} \;\vert\; \mathcal N \in \mathcal S_{\bm p}^{(\mathcal C)} \right\rbrace\right), \quad \theta \in R_{[0,1]}
	\end{align}
	i.e. the maximum normalized (MIDI-) velocity where $\mathcal N$ denotes each note within the corresponding slice $\mathcal S^{(\mathcal C)}_{\bm p}$, and $\bar{\theta} \in \mathbb R_{[0,1]}$ is the by the user requested normalized maximum velocity.
	
\subsection{Vertical Density} 
The Vertical Density scale action allows the user to control how many notes there ideally should be in the output. The parameters for $\gamma$ are given by
	\begin{align}
		\theta = \left\lvert \mathcal S^{(\mathcal C)}_{\bm p}\right\rvert, \quad \theta \in \mathbb Z_+	
	\end{align}
	and $\bar{\theta} \in \mathbb Z_+$ is the by the user requested number of notes.
	
\subsection{Duration}
The Duration scale action gives the user control over the average duration of each slice in the output. $\gamma$ is defined so that
	\begin{align}
		\theta = d_{\bm p}
	\end{align}
	where $d_{\bm p} \in \mathbb R_+$ denotes the duration in ticks of slice $\mathcal S^{(\mathcal C)}_{\bm p}$ and $\bar{\theta} \in \mathbb R_+$ is user-defined.


\subsection{Tempo Consistency}
When using a corpus consisting of multiple sections with varying tempi, one problem is that the tempo changes too often, sometimes with every slice, which may obscure the user's perception of beat. The Tempo Consistency scale action is designed to remedy this behaviour. Unlike the previous filters in this section, it is not designed so that the user defines a requested tempo (as this can already be done in the scheduler), rather it serves as a low-pass filter based on the history of previous tempi so that
	\begin{align}
		\theta &= \zeta_{\bm p}, \quad \theta \in \mathbb R_+ \\
		\bar{\theta} &= \frac{1}{q}\sum_{k=1}^q \zeta_{w-q}, \quad \bar{\theta} \in \mathbb R_+
	\end{align} 
	where $\zeta_{\bm p}$ denotes the tempo in BPM of slice $\mathcal S^{(\mathcal C)}_{\bm p}$, $\zeta_{w-q}$ denotes the tempo of previous output slice $\mathcal S^{(\mathcal Y)}_{w-q}$ and $q$ is a user controlled parameter corresponding to the length of the filter, i.e. how many previous output slices to take into account when applying the filter. 

\subsection{Spectral Distribution}
The final filter in this category, the Spectral Distribution scale action, is intended to give the user some control over the spectral distribution, which in this case corresponds to the number of MIDI notes in each octave, and is the only filter in this category that does not apply the above procedure of scaling based on a gaussian. Here $\gamma \in \mathbb R_{[0,1]}$ is defined as a function $\gamma \left( \bm{\theta}, \bm{\bar \theta}\right)$ where the user specifies a vector $\bm{\hat \theta} \in \mathbb R^{11}_{[0,1]}$ where the value at $\bm{\hat\theta}[i]$ corresponds to the weight to apply to the $i$:th octave, $i \in [0, 11)$.\footnote{Octave numbers start from MIDI note number 0, e.g. octave 0 corresponds to note numbers 0-11, octave 1 note numbers 12-23, etc., resulting in a total of 11 octaves, even if the bottom and top octaves are very rarely used in musical contexts.} A similar value $\bm{\theta} \in \mathbb R^{11}_{[0,1]}$ is calculated for each slice $\mathcal S^{(\mathcal C)}_{\bm p}$ corresponding to a peak using algorithm \ref{alg:spectraldist}:

\begin{algorithm}[h!]	
			\caption{Computing spectral distribution $\bm\theta$ for slice $\mathcal S^{(\mathcal S)}_{\bm p}$}
			\begin{spacing}{1.2}
			\begin{algorithmic}\label{alg:spectraldist}
				\STATE{$\bm\theta := \begin{bmatrix} 0 & \dots & 0\end{bmatrix}$}
				\FOR{$\mathcal N \in \mathcal S^{(\mathcal C)}_{\bm p}$}
					\STATE{$b := \left\lfloor \frac{\mathcal N\texttt{.pitch}}{12}\right\rfloor$}
					\STATE{$\bm\theta[b] :=  \bm\theta[b] + 1$}
				\ENDFOR
				\STATE{$\bm\theta := \left(\max\limits_{\theta \in \bm\theta}\theta\right)^{-1}\bm\theta$}
			\end{algorithmic}
			\end{spacing}
		\end{algorithm}
		
\noindent Finally, $\gamma$ is given by the Euclidean distance between the requested distribution and the actual, i.e.
	\begin{align}
		\gamma\left(\bm \theta, \bm{\bar \theta}\right) = \left\lVert \bm\theta - \bm{\bar \theta}\right\rVert_2.
	\end{align}


	


%%%%%%%%%%%%%%%%%%%%%%%%%
\section{Regional Masking}
The final scale action is the Regional Masking scale action, which allows the user to specify a region of indices within the corpus that the output should be selected from, e.g. to select only one of multiple movements in a multi-movement corpus. More specifically, the user specifies an interval $[a, b],\; a,b \in \mathbb Z_{[1,U]},\; a < b$ where $U$ denotes the size of corpus $\mathcal C$, and applies a parameter $\gamma$ (according to equation \ref{eq:filter-scaling}) defined as
	\begin{align}
		\gamma = \left\lbrace\begin{array}{ll}
			1 & \text{if } a \le u\left(\mathcal S^{(C)}_{\bm p}\right) \le b \\
			0 & \text{otherwise}.
		\end{array}\right.
	\end{align}

%%%%%%%%%%%%%%%%%%%%%%%%%
\section{Output Selection Modes}\label{ssec:output-sel}
The final step in the entire generation process is as described in section \ref{sec:3-generate} to select an output slice $\mathcal S^{(\mathcal Y)}_w$ from the scaled peak matrix $\bm P^{(\ast)}_w$ at time step $w$. This was in section \ref{sec:3-generate-output} defined as the slice corresponding to the peak with the highest score (or if no peaks exist, simply the slice in the corpus following the previous output slice), but can be generalized to a selection function $\Omega \colon \mathbb R^{3\times K} \rightarrow \mathcal S$ defined as
	\begin{align}\label{eq:outputsel}
		\mathcal S^{(\mathcal Y)}_w = \Omega\left(\bm P^{(\ast)}_w, \mathcal C, \mathcal H_w \right).
	\end{align}

This report introduces two new modes for output selection: Probabilistic output selection and Quality Threshold. To provide a better framework for the new classifiers, we will briefly revisit this default output selector and introduce some additional definitions, followed by the introduction of the new modes.

\subsection{Output Selection in the Generalized Case}\label{ssec:default-outputsel}
As per equation \ref{eq:outputsel}, we define the output slice $\mathcal S^{(\mathcal Y)}_w$ at time step $w$ as a function of the peak matrix $\bm P^{(\ast)}_w$, the corpus $\mathcal C$ and the history of previous output slices $\mathcal H_w$. More specifically, we can for all output selection modes define a set $\Xi = \left\lbrace \bm p_1, \dots, \bm p_H\right\rbrace$ of $H \in \mathbb Z_{\ast}$ viable peak candidates with the corresponding set of slices $\xi =\left\lbrace \mathcal S^{(\mathcal C)}_{\bm p_1}, \dots, \mathcal S^{(\mathcal C)}_{\bm p_H}\right\rbrace$. For each slice, a probability $\rho_h$ is defined so that
	\begin{align}
		\rho_h = P\left(\mathcal S^{(\mathcal C)}_{\bm p_h}\right), \quad \forall\, \mathcal S^{(\mathcal C)}_{\bm p_h} \in \xi.
	\end{align}
	Given a stochastic variable $X \sim \text{U}[0,1]$ the output slice is selected through a sampling function $g(\rho)$ so that
	\begin{align}\label{eq:samplingproba}
		g(\rho) = \left\lbrace\begin{array}{cl}
			\mathcal S^{(\mathcal C)}_{\bm p_1} & \text{if } X < \rho_1\\
			\mathcal S^{(\mathcal C)}_{\bm p_2} & \text{if } \rho_1 \le X < \rho_2\\
			\vdots\\
			\mathcal S^{(\mathcal C)}_{\bm p_{H-1}} & \text{if } \rho_{H-2} \le X < \rho_{h-1}\\
			\mathcal S^{(\mathcal C)}_{\bm p_H} & \text{otherwise}
		\end{array}\right.
	\end{align}
	
In the case of the default output selection mode («Maximum»), we define $\Xi$ as the set of peaks corresponding to the peak with the highest score ($H>1$ if multiple peaks have the same maximum value), i.e.
	\begin{align}
		\Xi = \left\lbrace \bm p \,\bigg\vert	\, \bm p \in \bm P^{(\ast)}_w \wedge \left( y_{\bm p} = \max_{\bm p_j \in \bm P^{(\ast)}_w} y_{\bm p_j}\right)\right\rbrace,
	\end{align}
	with corresponding probabilities
	\begin{align}\label{eq:max-rho}
		\rho_h = P\left(\mathcal S^{(\mathcal C)}_{\bm p_h}\right) = \frac{1}{\left\lvert \Xi \right\rvert}, \quad \forall \,\mathcal S^{(\mathcal C)}_{\bm p_h} \in \xi.
	\end{align}
	The output slice $\mathcal S^{(\mathcal Y)}_w$ is given by
	\begin{align}\label{eq:defaultpeaksel}
		\Omega\left(\bm P^{(\ast)}_w, \mathcal C, \mathcal H_w\right)
		= \left\lbrace\begin{array}{ll}
			g(\rho) & \text{if } \Xi \ne \varnothing\\
			\mathcal S^{(\mathcal C)}_{u+1} & \text{otherwise},
		\end{array}\right.
	\end{align}
	where $g(\rho)$ is defined according to equation \ref{eq:samplingproba} and $\mathcal S^{(\mathcal C)}_{u+1}$ denotes the slice following the previous output slice, i.e. the slice $\mathcal S^{(\mathcal C)}\in \mathcal C$ fulfilling the condition
	\begin{align}\label{eq:defaultpeaksel-emptycase}
		u\left(\mathcal S^{(\mathcal C)}\right) = u\left(\mathcal S^{(\mathcal Y)}_{w-1}\right) + 1 \qquad \text{mod}\, U.
	\end{align}
	

%%
\subsection{Probabilistic Output Selection}\label{ssec:proba-output-sel}
The Probabilistic output selection mode will, unlike the default output selection mode, take all peaks into account and create a probability density function where the probability is defined by the score of each peak, i.e.
	\begin{align}
		\Xi = \left\lbrace \bm p \, \bigg\vert \, \bm p \in \bm P^{(\ast)}_w\right\rbrace
	\end{align}
	and
	\begin{align}\label{eq:proba-rho}
		\rho_h = P\left(\mathcal S^{(\mathcal C)}_{\bm p_h}\right) = \frac{y_{\bm p_h}}{\sum_{i = 1}^H y_{\bm p_i}},
	\end{align}
	where $g(\rho)$ and $\Omega$ are defined according to equations \ref{eq:samplingproba} and \ref{eq:defaultpeaksel} respectively. In this mode, the probability of matches with lower quality (i.e. peaks with a lower score) increases, but the variation increases. Due to the latter, it's less likely to get stuck in deterministic loops when interacting with the system.

%%%%%%%%%%%%%%%%%%%%%%%%%
\section{Quality Threshold}\label{ssec:qual-thresh}
The Quality Theshold output selection mode is an extension that can be combined with either of the two previous modes, in which the user defines a threshold $\psi \in \mathbb R_+$, where all peaks with scores below this value will be discarded, i.e.
	\begin{align}
		\Xi = \left\lbrace \bm p \,\bigg\vert	\, \bm p \in \bm P^{(\ast)}_w \wedge \left( y_{\bm p} = \max_{\bm p_j \in \bm P^{(\ast)}_w} y_{\bm p_j}\right)\right\rbrace
		\cap \left\lbrace \bm p \, \bigg\vert \, \bm p \in \bm P^{(\ast)}_w \wedge y_{\bm p} \ge \psi\right\rbrace
	\end{align}
	when combined with the default («Maximum») mode and 
	\begin{align}
		\Xi = \left\lbrace \bm p \, \bigg\vert \, \bm p \in \bm P^{(\ast)}_w\right\rbrace 
		\cap \left\lbrace \bm p \, \bigg\vert \, \bm p \in \bm P^{(\ast)}_w \wedge y_{\bm p} \ge \psi\right\rbrace
	\end{align}
	when combined with the probabilistic mode. The output selection function is in both cases defined as
	\begin{align}\label{eq:threshold-output}
		\Omega\left(\bm P^{(\ast)}_w, \mathcal C, \mathcal H_w\right) = \left\lbrace\begin{array}{ll}
				g(\rho) & \text{if } \Xi \ne \varnothing\\
				\text{undefined} & \text{otherwise},
			\end{array}\right.
	\end{align}
	where $\rho$ is defined as equations \ref{eq:max-rho} and \ref{eq:proba-rho} respectively and $g(\rho)$ as equation \ref{eq:samplingproba}. More importantly, as seen in equation \ref{eq:threshold-output}, the output is undefined if no peaks exists in $\Xi$, either because no matches were found in earlier stages of the generation process so that $\bm P^{(\ast)}_w$ is empty or because no peaks fulfil the condition $y_{\bm p} \ge \psi$. When undefined, no output will be generated for this particular generation step $w$. In other words, if the quality of the slices corresponding to the existing peaks does not match the requested quality (as defined by the threshold $\psi$), no output will be generated at all, hence introducing a degree of sparsity in the generated material.




