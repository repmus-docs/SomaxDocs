\chapter{Class Diagram}\label{sec:a-class-diagram}

Here, a class diagram for the entire Python architecture is presented. A high resolution image is also available in the GitHub repository\footnote{\url{https://github.com/repmus-docs/SomaxDocs/tree/master/UML}}. Note that final classes that extend a particular abstract class are not represented in this diagram (e.g. the \texttt{NextStateScaleAction} or any of the other 12 scale actions that implement the \texttt{AbstractScaleAction} interface), as this would clutter the diagram and significantly reduce the readability. Non-final classes that extend an abstract class are still present as they are important from an architectural perspective (e.g. \texttt{RendererEvent} which extends the abstract \texttt{ScheduledEvent}). 

Finally, certain aspects of the architecture have been omitted as they currently do not serve any meaningful architectural purpose but are simply present for legacy reasons (e.g. the distinction between the classes \texttt{Agent} and \texttt{OscAgent}, where the former has been omitted since it no longer serves any purpose), or stateless/static classes that serve little purpose from an architectural point of view (e.g. enum classes). 

 \begin{figure}[h!]
    \centering        
 	\includegraphics[width=1.4\textwidth, keepaspectratio, angle=90]{figures/class-diagram.png}
    \caption{Class Diagram for the Python architecture. Note: a high resolution image is available in the GitHub repository.}
    \label{fig:a-class-diagram}
\end{figure}